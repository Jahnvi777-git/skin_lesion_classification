\chapter{CHAPTER TITLE} \label{c1}
\justifying
\section {Introduction}
A thesis or dissertation is a document submitted in support of candidature for an academic degree or professional qualification presenting the author's research and findings. In some contexts, the word "thesis" or a cognate is used for part of a bachelor's or master's course, while "dissertation" is normally applied to a doctorate, while in other contexts, the reverse is true.

\gls{FS} These guidelines are provided to formally expose you to the various ethical and technical issues involved in writing up your work and the format you are required to adhere to while submitting your work as Ph.D / M.Tech [By Research] Synopsis / Thesis or M.Phil dissertation. 

\section {Ethics involved}

 Knowing the difference between ethical and unethical practices in technical writing requires an understanding of plagiarism, paraphrasing, and quotation. These concepts are defined below. The definitions are reproduced from the `Handbook of Technical Writing' by Brusaw.

\section{Plagiarism}

To use someone else's exact words without quotation marks and appropriate credit, or to use the unique ideas of someone else without acknowledgement, is known as plagiarism. In publishing, plagiarism is illegal; in other circumstances, it is, at the least, unethical. You may quote or paraphrase the words or ideas of another if you document your source. Although you need not enclose the paraphrased material in quotation marks, you must document the source. 


Paraphrased ideas are taken from someone else whether or not the words are identical. Paraphrasing a passage without citing the source is permissible only when the information paraphrased is common knowledge in a field. (Common knowledge refers to historical, scientific, geographical, technical, and other type of information on a topic readily available in handbooks, manuals, atlases and other references) \gls{URL}. 
