\chapter{CHAPTER TITLE} \label{c2}
\section {Review of literature}
A literature review is a text of a scholarly paper, which includes the current knowledge including substantive findings, as well as theoretical and methodological contributions to a particular topic \cite{slavin2014medical}. Literature reviews are secondary sources, and do not report new or original experimental work. Most often associated with academic-oriented literature, such reviews are found in academic journals, and are not to be confused with book reviews that may also appear in the same publication. Literature reviews are a basis for research in nearly every academic field. A narrow-scope literature review may be included as part of a peer-reviewed journal article presenting new research, serving to situate the current study within the body of the relevant literature and to provide context for the reader. In such a case, the review usually precedes the methodology and results sections of the work.

Producing a literature review may also be part of graduate and post-graduate student work, including in the preparation of a thesis, dissertation, or a journal article. Literature reviews are also common in a research proposal or prospectus (the document that is approved before a student formally begins a dissertation or thesis).

\section {Review types}
The main types of literature reviews are: evaluative, exploratory, and instrumental.

A fourth type, the systematic review, is often classified separately, but is essentially a literature review focused on a research question, trying to identify, appraise, select and synthesize all high-quality research evidence and arguments relevant to that question. A meta-analysis is typically a systematic review using statistical methods to effectively combine the data used on all selected studies to produce a more reliable result.


\subsection {Process and product}
 distinguish between the process of reviewing the literature and a finished work or product known as a literature review. The process of reviewing the literature is often ongoing and informs many aspects of the empirical research project. All of the latest literature should inform a research project. Scholars need to be scanning the literature long after a formal literature review product appears to be completed.

\section {Page limitation}
A careful literature review is usually 15 to 30 pages and could be longer. The process of reviewing the literature requires different kinds of activities and ways of thinking and link the activities of doing a literature review with Benjamin Bloom’s revised taxonomy of the cognitive domain (ways of thinking: remembering, understanding, applying, analysing, evaluating, and creating).
