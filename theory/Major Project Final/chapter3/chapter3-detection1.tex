\chapter{CHAPTER TITLE} \label{c3}
\section{Paraphrasing}
When you paraphrase a written passage, you rewrite it to state the essential ideas in your own words. Because you do not quote your source word for word when paraphrasing, it is unnecessary to enclose the paraphrased material in quotation marks. However, the paraphrased material must be properly referenced because the ideas are taken from someone else whether or not the words are identical. 

Ordinarily, the majority of the notes you take during the research phase of writing your report will paraphrase the original material. Paraphrase only the essential ideas. Strive to put original ideas into your own words without distorting them."

\section{Quotations}
When you have borrowed words, facts, or idea of any kind from someone else's work, acknowledge your debt by giving your source credit in footnote (or in running text as cited reference). Otherwise, you will be guilty of plagiarism. Also, be sure you have represented the original material honestly and accurately. Direct word to word quotations are enclosed in quotation marks."

When you use programs written by others with or without modifications, the report/thesis must clearly bring this out with proper references, and must also reflect the extent of modification introduced by you, if any. A modified program is not entirely yours. Only a program, which you write from scratch, does not require source to be identified. Identification of source in all other cases is must. Standard subroutines (even if public domain) used in your programs must be properly referenced. Although programs need not be appended to the thesis, they must be submitted to your research supervisor in hard copy and other media. Inclusion of a computational flow chart in your thesis is highly recommended, however. 

\gls{SVM}The material presented in the thesis/report must be self contained. A reader must be able to reproduce your experimental, theoretical, computational, and simulations results based on the information presented in the thesis. You must mention the names of the suppliers whose chemicals/instruments were used in the work to allow a reader to setup an experiment. While discussing issues related to computation time, the hardware used must be specified accurately, using processor speed, etc \gls{SVM}. 


\subsection{Quotation and reference to earlier work}

\gls{HTML} If reproduction of some text material available in a published work can enhance the value to your thesis, you can add it to your thesis in the form of quoted material or a quotation. Such material should be indented on both sides over and above the indentation used for the regular text. It should preferably be single spaced, and appear as a separate paragraph(s), whether short or long. The idea is to make such material stand out from the rest of the text that you have written. Clearly, too many quotations or quoted paragraphs are not desirable in a thesis which is an original piece of work. Not quoting a material taken verbatim from another source is however plagiarism. Paraphrasing and giving credit to the author(s) is more accepted way of referring to earlier works \gls{HTML}. 

\subsection {Use of abbreviations}
In Chapters, while introducing abbreviations, follow:

\begin{verbatim}
\newacronym{URL}{URL}{Uniform Resource Locator}
\end{verbatim}

and in chapters it can be used as:

\begin{verbatim}
\gls{URL}
\end{verbatim}

Which produces the following output first time when you call it: \\ \gls{URL} and simply \gls{URL} each subsequent time.

\section {References}
Choose a respected journal in your field in which title of the paper also appear in the list of references and consistently follow the citation style used by this journal. Names of all the authors with their initials, title of the article, names of editors for edited books or proceedings, and the range of pages that contain the referenced material must appear in the bibliography. You should not mix citation styles of several journals and not to create your own style. 

\subsection {Citation format}
All references and citation should be of the standard "Harvard Style" (Author, Year)’ format.
\subsubsection {Single author citation}
\begin{itemize}
	\item Citation at the beginning of the sentence
	\begin{itemize}
		\item Jones (2011) emphasized that citations in a text should be consistent.
	\end{itemize}
	\item Citation at the End of the Sentence:
	\begin{itemize}
		\item It was emphasized that citations in a text should be consistent (Jones, 2011).
	\end{itemize}	
\end{itemize}

\subsubsection {Double authors citation}
\begin{itemize}
	\item Citation at the beginning of the sentence
	\begin{itemize}
		\item Jones and Baker (2011) emphasized that citations in a text should be consistent.
	\end{itemize}
	\item Citation at the End of the Sentence
	\begin{itemize}
		\item It was emphasized that citations in a text should be consistent (Jones and Baker, 2011).
	\end{itemize}	
\end{itemize}

\subsubsection {More than two authors citation}
\begin{itemize}
	\item Citation at the beginning of the sentence
	\begin{itemize}
		\item Jones et al. (2011) emphasized that citations in a text should be consistent.
	\end{itemize}
	\item Citation at the End of the Sentence
	\begin{itemize}
		\item It was emphasised that citations in a text should be consistent (Jones et al., 2011).
	\end{itemize}	
\end{itemize}

\subsubsection {Sources written in the same year by the same author(s)}
\begin{itemize}
	\item It was emphasized that citations in a text should be consistent (Jones, 1998a). In a work published later that year, Jones (1998b) proposed that...
	\item It was emphasized that citations in a text should be consistent (Jones, 1998a; 1998b).
\end{itemize}

\subsubsection {Sources written by same author(s) in different Year(s)}
\begin{itemize}
	\item (Smith, 2013; 2005; 2001)
\end{itemize}



